%%%%%%%%%%%%%%%%%%%%%%%%%%%%%%%%%%%%%%%%%
% Original author:
% Linux and Unix Users Group at Virginia Tech Wiki
% (https://vtluug.org/wiki/Example_LaTeX_chem_lab_report)
% Modified by: Hector F. Jimenez S, for the Digital Electronics Laboratory.
% License:
% CC BY-NC-SA 3.0 
%%%%%%%%%%%%%%%%%%%%%%%%%%%%%%%%%%%%%%%%%
%----------------------------------------
%	PACKAGES AND DOCUMENT CONFIGURATIONS
%---------------------------------------

\documentclass[paper=a4, fontsize=12pt]{article} 		% A4 paper and 11pt font size
\usepackage[T1]{fontenc} 								% Use 8-bit encoding that has 256 glyphs
%\usepackage{fourier}		 							% Use the Adobe Utopia font for the document 
\usepackage[spanish,english]{babel}						% Spanish Language, templates uses some sections in english.
\selectlanguage{spanish}								% main language.
\PassOptionsToPackage{spanish}{babel}
%\renewcommand{\figurename}{Figura}						% Force rename of figure.
%\renewcommand{\figurename}{Fig.}
\usepackage[figurename=Fig.]{caption}
\usepackage[utf8]{inputenc}								% tildes for spanish language.
\usepackage{amsmath,amsfonts,amsthm} 					% Math packages.
\usepackage{minted}										% For syntax highlighting.
\usepackage{float}										% Image will be in the same place as you want.!!! x-/
\usepackage{sectsty} 									% Allows customizing section commands
\allsectionsfont{\centering \normalfont\scshape}	   	% Make all sections centered, the default font and small caps
\usepackage{hyperref}
\hypersetup{											%Setups the false color and borders.
    colorlinks=false,
    pdfborder={0 0 0},
}
\newcommand\fnurl[2]{%									% set a simple and quick footnote command and include url.
\href{#2}{#1}\footnote{\url{#2}}%	
}
\usepackage{graphicx}									% Import easyly images.
\graphicspath{ {./images/} }							% Where to look for the images.
\DeclareGraphicsExtensions{.pdf,.png,.jpg}				% Graphics Extension to be used
\usepackage[notes,backend=biber]{biblatex-chicago}		% Bibliography and references.
\bibliography{biblio}									% bibliography filename.
\usepackage{fancyhdr} 									% Custom headers and footers
\pagestyle{fancyplain} 									% Makes all pages in the document conform to the custom headers and footers
\fancyhead{} 											% No page header
\fancyfoot[L]{} 										% Empty left footer
\fancyfoot[C]{} 										% Empty center footer
\fancyfoot[R]{\thepage} 								% Page numbering for right footer
\renewcommand{\headrulewidth}{0pt} 						% Remove header underlines
\renewcommand{\footrulewidth}{0pt} 						% Remove footer underlines
\setlength{\headheight}{13.6pt} 					    % Customize the height of the header
\numberwithin{equation}{section}						% Number equations within sections (i.e. 1.1, 1.2, 2.1, 2.2 instead of 1, 2, 3, 4)
%\numberwithin{figure}{section} 						% Number figures within sections (i.e. 1.1, 1.2, 2.1, 2.2 instead of 1, 2)
\numberwithin{table}{section} 							% Number tables within sections (i.e. 1.1, 1.2, 2.1, 2.2 instead of 1, 2, 3, 4)
\setlength\parindent{0pt} 								% Removes all indentation from paragraphs

\usepackage{listings}									% http://ctan.org/pkg/listings
\renewcommand{\lstlistingname}{Codigo}	

%\newcommand{\horrule}[1]{\rule{\linewidth}{#1}} 		% Create horizontal rule command with 1 argument of height
%%%%%%%%%%%%%%%%%%%%
%Title Section
%%%%%%%%%%%%%%%%%%%%%
\title{Desarrollo de un Turnero Digital\\ 
Usando FPGA's \\
Laboratorio de Electrónica Digital\\Módulo: 4, Aplicación de Escritorio} 			% Title
%\horrule{0.5pt} \\[0.4cm] 								% Thin top horizontal rule	Title rule
%\huge Assignment Title \\ 								% The assignment title
%\horrule{2pt} \\[0.5cm] 								% Thick bottom horizontal rule
\author{												% Authors begin.
Héctor F. \textsc{Jiménez S.}\\
\texttt{hfjimenez@utp.edu.co} \\
\texttt{PGP KEY ID: 0xB05AD7B8}
\and
Sebastian \textsc{Zapata}\\
\texttt{sebastzapata93@gmail.com }\\
\texttt{PGP KEY ID: 0xfffff}
\and 
Francisco \textsc{Gallego}\\
\texttt{juanfran16@utp.edu.co}
} 												       % End of  Author name
\date{}    						                       % Date for the report, this will hide the \today.

\begin{document}
\maketitle                      			           % Insert the title, author and date
\begin{center}
\begin{tabular}{l r}								   % two column to
Fecha de Entrega: & \textbf{16} noviembre,2016 \\				   % Ramiro's Details.
Profesor: & Ing.Msc(c) Ramiro Andres Barrios Valencia
\end{tabular}
\end{center}
%%%%%%%%%%%	
% Let's start the document.
%%%%%%%%%%%	
\section{Objetivos}
\begin{itemize}
  \item Desarrollar el aplicativo de referencia Turno-Puesto de atención.
\end{itemize}
$\\$
%%%%%%%%%%%	
% Theory Marc! 
%%%%%%%%%%%	
\section{Marco Teórico}

En este modulo desarrollamos el aplicativo utilizando como lenguaje de programación \emph{python} que posee soporte para el manejo de comunicaciones utilizando el protocolo rs232, la librería estándar se llama \textbf{\textit{pyserial}}, específicamente para nuestro desarrollado utilizamos la siguientes opciones:
\begin{enumerate}
\item 8 bits \textbf{serial.EIGHTBITS}
\item 2 bits de parada \textbf{serial.STOPBITS\_TWO}
\item paridad impar \textbf{serial.PARITY\_ODD}
\item velocidad de \textbf{9600} baudios.
\end{enumerate}

para nosotros simular el envió de comunicación serial lo realizamos en dos sistemas operativos para windows  utilizamos un emulador de puertos \emph{Virtual Serial Port en modo pair} que nos permite crear virtualmente puertos/dispositivo que utilizara el protocolo RS232. 
En Gnu-Linux utilizamos \textbf{socat}, un emulador multipropuesta que permite crear dispositivos de comunicaciones virtuales unidireccionales.
la forma de crear dos dispositivos en Linux es :
\begin{listing}[H]
	\begin{minted}{python}
socat -d -d  pty,raw,echo=0 pty,raw,echo=0
\end{minted}
\caption{Creacion de dos dispositivos ubicados en \textbf{/dev/pts/2, /dev/pts/3}.}
    \label{socats}
\end{listing}

En este sistema se debe agregar las siguientes librerias a pyserial para poder abrir y acceder al puerto serial : \textbf{rtscts=True,dsrdtr=True}

La idea principal de este modulo era simular como seria la integracion entre el hardware y el aplicativo que se desarrollo, nosotros el aplicativo lo dividimos principalmente en 3 modulos, escritos en 3 scripts diferentes de python  que son : 
\begin{enumerate}
\item \textbf{Generador de Turnos}: Es un script en python que se encarga de generar los turnos solicitados por el usuario
\item \textbf{ModuloA,B}: Es es el aplicativo al que tienen acesso los asesores o mesas que resuelve el turno. 
\item \textbf{Visualizador de Turnos(\textit{Turnero})}: Es el que se encarga de mostrar el turno actual en pantalla para el publico.
\end{enumerate}
Para la interfaz gráfica se utilizo las librerias que dispone python como \textbf{pygame} el cual nos permite de una manera simple y rápida crear ventanas y entornos.
\begin{figure}[H]
  \centering
     \includegraphics[scale=0.38]{imgs/aplicativo.png}
  \caption{Esquema de Conexión Aplicación Web.}
    \label{fig:senal}
\end{figure}

\begin{listing}[H]
	\begin{minted}{python}
# -*- coding: utf-8 -*-			
import pygame		#Importar libreria necesaria
import serial
ANCHO=400			#Dimensiones de la ventana
ALTO=400	
BLANCO=(255,255,255)#constante de colores
ROJO=(255,0,0)
VERDE=(0,255,255)
AZUL=(0,0,255)
NEGRO=(0,0,0)

pygame.init()		#objeto que inicializa el entonro	
pantalla=pygame.display.set_mode([ANCHO,ALTO])#creamos ventana
pygame.display.set_caption("Generador De Turnos") 
\end{minted}
\caption{Creación  de ventanas en todos los scripts.}
    \label{window}
\end{listing}
Teniendo en cuenta esto para cada script lo que nosotros hemos realizado en nuestra aplicación simulada es que el usuario una vez presione la tecla \textbf{n}  utilizamos la opción de eventos que ofrece pygame y enviamos esto por serial a nuestro visualizador. 

\begin{listing}[H]
	\begin{minted}{python}
	for event in pygame.event.get():
			if event.type == pygame.QUIT:
				fin=True
			elif event.type ==	pygame.KEYUP:
				if event.key == pygame.K_n:
					if activado==0:
						#serial
						x=s.write(turn)
						print ("Envio : -- nuevo turno ")
						#pygame
						activado=1
						turnos+=1
						if turnos==100:
							turnos=1
\end{minted}
\caption{Gestión de eventos, presión de teclas .}
    \label{presion}
\end{listing}

Una vez se detecta el evento de la tecla enviamos esto el numero d eturno generado. 

\section{Anexos}
los demás archivos son adicionales que resuelven la misma implementación, se encuentran en el \fnurl{repositorio}{https://github.com/heticor915/DigitalElectronicsLab/tree/master/Reports/Module4}
\textit{Nota}: Las referencias utilizadas se encuentran en los pies de página. Si requiere de manera detallada estas contacte con \emph{miembros del equipo.}
\end{document}
